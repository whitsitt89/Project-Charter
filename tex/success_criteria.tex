There are several checkpoints that this project will need to accomplish to be a success.  The first and foremost is that the hardware components must function to accurately measure the specific gravity and temperature of the fermenting liquid.  The components must also be watertight and food safe.  The hydrometer must remain submerged in an enclosed environment for at least two weeks.  The hydrometer must be able to with stand temperatures of up to 90 degrees Fahrenheit for extended amounts of time.  The hydrometer must have at least a two week battery life.

The blue tooth hydrometer must interact with a local server set up on a raspberry pi 3.  This server must take the analog information from the hydrometer, convert it to digital format, and then store the information.  The server must check for irregularities, and if found send a request to the hydrometer to remeasure.  

The mobile web interface must accurately and easily display all information needed by a user.  The user must be able to see the same information in a similar, if not the same, format whether on an Android or IOS product.  All data is to be stored onto the local server for records.